% --------------------------------------------------------------------------- %
% Poster for the ECCS 2011 Conference about Elementary Dynamic Networks.      %
% --------------------------------------------------------------------------- %
% Created with Brian Amberg's LaTeX Poster Template. Please refer for the     %
% attached README.md file for the details how to compile with `pdflatex`.     %
% --------------------------------------------------------------------------- %
% $LastChangedDate:: 2011-09-11 10:57:12 +0200 (V, 11 szept. 2011)          $ %
% $LastChangedRevision:: 128                                                $ %
% $LastChangedBy:: rlegendi                                                 $ %
% $Id:: poster.tex 128 2011-09-11 08:57:12Z rlegendi                        $ %
% --------------------------------------------------------------------------- %
\documentclass[a0paper,portrait]{baposter}
\usepackage[utf8x]{inputenc}

\usepackage{amsmath}
\usepackage{amsthm}
\usepackage{amssymb}
\usepackage{shuffle}
\usepackage{tikz}
\usetikzlibrary{shapes}
\usetikzlibrary{snakes}


\usepackage{graphicx,wrapfig,lipsum}
\usepackage{hyperref}
\usepackage{cleveref}
\crefname{coro}{Corollary}{Corollaries}
\crefname{defi}{Definition}{Definitions}
\crefname{lem}{Lemma}{Lemmas}
\crefname{algocf}{Algorithm}{Algorithms}
\crefname{ex}{Example}{Examples}
\crefname{rem}{Remark}{Remarks}
\newtheorem{theorem}{Theorem}
\newtheorem{lem}[theorem]{Lemma}
\newtheorem{conj}[theorem]{Conjecture}
\newtheorem{coro}[theorem]{Corollary}
\newtheorem{prop}[theorem]{Proposition}
\newtheorem{prob}{Problem}

\theoremstyle{definition}
\newtheorem{defi}[theorem]{Definition}
\newtheorem{fact}[theorem]{Fact}
\theoremstyle{remark}
\newtheorem{ex}[theorem]{Example}
\newtheorem{rem}[theorem]{Remark}
\newtheorem{nota}[theorem]{Notations}

\usepackage{relsize}		% For \smaller
\usepackage{url}			% For \url
% \usepackage{epstopdf}	% Included EPS files automatically converted to PDF to include with pdflatex

%%% Global Settings %%%%%%%%%%%%%%%%%%%%%%%%%%%%%%%%%%%%%%%%%%%%%%%%%%%%%%%%%%%

\graphicspath{{pix/}}	% Root directory of the pictures 
% \tracingstats=2			% Enabled LaTeX logging with conditionals

%%% Color Definitions %%%%%%%%%%%%%%%%%%%%%%%%%%%%%%%%%%%%%%%%%%%%%%%%%%%%%%%%%

\definecolor{bordercol}{RGB}{40,40,40}
\definecolor{headercol1}{RGB}{ 195, 222, 234 }% 177, 202, 213 }%  169, 206, 223 }%129, 179, 201}%186,215,230}
\definecolor{headercol2}{RGB}{  195, 222, 234 }%177, 202, 213 }% 169, 206, 223 }%129, 179, 201}%186,215,230}
\definecolor{headerfontcol}{RGB}{0,0,0}
\definecolor{boxcolor}{RGB}{255,255,255}

%%%%%%%%%%%%%%%%%%%%%%%%%%%%%%%%%%%%%%%%%%%%%%%%%%%%%%%%%%%%%%%%%%%%%%%%%%%%%%%%
%%% Utility functions %%%%%%%%%%%%%%%%%%%%%%%%%%%%%%%%%%%%%%%%%%%%%%%%%%%%%%%%%%

%%% Save space in lists. Use this after the opening of the list %%%%%%%%%%%%%%%%
\newcommand{\compresslist}{
  \setlength{\itemsep}{1pt}
  \setlength{\parskip}{0pt}
  \setlength{\parsep}{0pt}
}

\include{newcommands}

%%%%%%%%%%%%%%%%%%%%%%%%%%%%%%%%%%%%%%%%%%%%%%%%%%%%%%%%%%%%%%%%%%%%%%%%%%%%%%%
%%% Document Start %%%%%%%%%%%%%%%%%%%%%%%%%%%%%%%%%%%%%%%%%%%%%%%%%%%%%%%%%%%%
%%%%%%%%%%%%%%%%%%%%%%%%%%%%%%%%%%%%%%%%%%%%%%%%%%%%%%%%%%%%%%%%%%%%%%%%%%%%%%%

\begin{document}
\typeout{Poster rendering started}

%%% Setting Background Image %%%%%%%%%%%%%%%%%%%%%%%%%%%%%%%%%%%%%%%%%%%%%%%%%%
\background{
	\begin{tikzpicture}[remember picture,overlay]%
	\draw (current page.north west)+(-2em,2em) node[anchor=north west]
	{\includegraphics[height=1.1\textheight]{fond-gris-degrade-clair.jpg}};
	\end{tikzpicture}
}

%%% General Poster Settings %%%%%%%%%%%%%%%%%%%%%%%%%%%%%%%%%%%%%%%%%%%%%%%%%%%
%%%%%% Eye Catcher, Title, Authors and University Images %%%%%%%%%%%%%%%%%%%%%%
\begin{poster}{
	grid=false,
	% Option is left on true though the eyecatcher is not used. The reason is
	% that we have a bit nicer looking title and author formatting in the headercol
	% this way
	% eyecatcher=false,
        linewidth = 1pt,
	borderColor=bordercol,
	headerColorOne=headercol1,
	headerColorTwo=headercol2,
	headerFontColor=headerfontcol,
	% Only simple background color used, no shading, so boxColorTwo isn't necessary
	boxColorOne=boxcolor,
	headershape=rectangle,%roundedright,
	headerfont=\Large\sf\bf,
	textborder=rectangle,
	background=user,
	headerborder=open,
  boxshade=plain
}
%%% Eye Cacther %%%%%%%%%%%%%%%%%%%%%%%%%%%%%%%%%%%%%%%%%%%%%%%%%%%%%%%%%%%%%%%
{
	Eye Catcher, empty if option eyecatcher=false - unused
}
%%% Title %%%%%%%%%%%%%%%%%%%%%%%%%%%%%%%%%%%%%%%%%%%%%%%%%%%%%%%%%%%%%%%%%%%%%
{\sf\bf
	Une construction combinatoire d'un automorphisme bidendriphorme de WQSym
}
%%% Authors %%%%%%%%%%%%%%%%%%%%%%%%%%%%%%%%%%%%%%%%%%%%%%%%%%%%%%%%%%%%%%%%%%%
{
	\vspace{1em} Journées Nationales du GDR IM 2022\\ Hugo Mlodecki, \,
	{\smaller hugo.mlodecki@gmail.com}
}
%%% Logo %%%%%%%%%%%%%%%%%%%%%%%%%%%%%%%%%%%%%%%%%%%%%%%%%%%%%%%%%%%%%%%%%%%%%%
{
% The logos are compressed a bit into a simple box to make them smaller on the result
% (Wasn't able to find any bigger of them.)
\setlength\fboxsep{0pt}
\setlength\fboxrule{0.5pt}
	\fbox{
		\begin{minipage}{14em}
			\includegraphics[width=12.3em,height=5em]{logo-lisn.png}
			% \includegraphics[width=4em,height=4em]{elte_logo} \\
			% \includegraphics[width=10em,height=4em]{dynanets_logo}
			% \includegraphics[width=4em,height=4em]{aitia_logo}
		\end{minipage}
	}
}

\headerbox{Motivation}{name=intro,column=0,row=0}{
  \vspace{-0.6em} 
  \include{intro}
  \vspace{-0.6em}
}
\headerbox{Mots tassés}{name=pw,column=0,below=intro}{
  \vspace{-0.6em} 
  \include{pw}
}
\headerbox{Descentes globales}{name=gd,column=0,below=pw}{
  \vspace{-0.6em} 
  \include{gd}
}
\headerbox{Nouvelles décompositions}{name=ins,column=0,below=gd, above=bottom}{
  \vspace{-0.6em} 
  \include{ins}
}
\headerbox{Arbres biplans}{name=box_biplan,column=1, row=0}{
  \vspace{-0.6em} 
  \include{box_biplan}
}
\headerbox{Une involution sur les mots tassés}{name=involution,span=2,column=1,below=box_biplan}{
  \vspace{-0.6em} 
  \include{involution}
}
\headerbox{Bases de $\WQSym$ et $\WQSym^*$}{name=bases,column=2,row=0, above=involution}{
  \vspace{-0.6em} 
  \include{bases}
}
\headerbox{Ouvertures}{name=ouvertures,column=1,below=involution,above=bottom}{
  \vspace{-0.6em} 
  \include{ouvertures}
}
\headerbox{Références}{name=references,column=2,below=involution}{
  \include{references}
}
\headerbox{Remerciements}{name=acknowledgements,column=2,below=references,above=bottom}{
  \vspace{-0.6em} 
  \include{acknowledgements}
}
\end{poster}
\end{document}
